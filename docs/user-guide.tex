%%%%%
%%%%% Based on a design by Alvaro Armenteros and a template by Esa Hyttiä
%%%%%

\documentclass[a4paper,twoside,11pt]{report}

%%%%%%%%%%%%%%%%%%%%%%%%%%%%%%%%%%%%%%%%%%%%%%%%%%%%%%%%%%%%%%%%%%%%%%%%%%%%%%%%%%
% Template Element Definitions
% Change these for each deliverable
%

\newcommand{\deliverabletitle}{User Guide for mmb 0.3.2}
\newcommand{\deliverabletitleshort}{mmb 0.3.2}

\newcommand{\deliverableauthorlist}{
& K.Edeline, J.Iurman \\
}

%%%%%%%%%%%%%%%%%%%%%%%%%%%%%%%%%%%%%%%%%%%%%%%%%%%%%%%%%%%%%%%%%%%%%%%%%%%%%%
%%%%%%%%%%%%%%%%%%%%%%%%%%%%%%%%%%%%%%%%%%%%%%%%%%%%%%%%%%%%%%%%%%%%%%%%%%%%%%
%% LATEX MACHINERY BEGINS HERE - NO USER-SERVICEABLE PARTS INSIDE
%%%%%%%%%%%%%%%%%%%%%%%%%%%%%%%%%%%%%%%%%%%%%%%%%%%%%%%%%%%%%%%%%%%%%%%%%%%%%%
%%%%%%%%%%%%%%%%%%%%%%%%%%%%%%%%%%%%%%%%%%%%%%%%%%%%%%%%%%%%%%%%%%%%%%%%%%%%%%

%%%%%%%%%%%%%%%%%%%%%%%%%%%%%%%%%%%%%%%%%%%%%%%%%%%%%%%%%%%%%%%%%%%%%%%%%%%%%%
% Document geometry
%

\setlength{\textheight}{230mm}
\setlength{\textwidth}{160mm}
\setlength{\voffset}{-25mm}
\setlength{\oddsidemargin}{0mm}
\setlength{\evensidemargin}{0mm}
\addtolength{\parskip}{0.33\baselineskip}

%%%%%%%%%%%%%%%%%%%%%%%%%%%%%%%%%%%%%%%%%%%%%%%%%%%%%%%%%%%%%%%%%%%%%%%%%%%%%%
% Package definitions
%

\usepackage{helvet}
\usepackage{graphicx}
\usepackage{subfigure}
\usepackage{color}
\usepackage{xcolor}
\usepackage{chngcntr}
\usepackage{longtable}
\usepackage{lastpage}
\usepackage{fancyhdr}
\usepackage{setspace}
\usepackage{url}
\usepackage{amsmath}
\usepackage{acronym}
\usepackage{listings}
\usepackage{balance}
\usepackage{array}
\usepackage{hyperref}

%%%%%%%%%%%%%%%%%%%%%%%%%%%%%%%%%%%%%%%%%%%%%%%%%%%%%%%%%%%%%%%%%%%%%%%%%%%%%%
% Font definitions
%
\renewcommand\familydefault{\sfdefault}

%%%%%%%%%%%%%%%%%%%%%%%%%%%%%%%%%%%%%%%%%%%%%%%%%%%%%%%%%%%%%%%%%%%%%%%%%%%%%%
% Heading style definitions
%

\setcounter{secnumdepth}{4}

\makeatletter
\renewcommand{\chapter}{\@startsection{chapter}{0}{0mm}
  {\baselineskip}%
  {\baselineskip}{\clearpage\LARGE\bf\color{black}}}
\renewcommand{\section}{\@startsection{section}{1}{0mm}
  {\baselineskip}%
  {\baselineskip}{\LARGE\color{black}}}%
\renewcommand{\subsection}{\@startsection{subsection}{2}{0mm}
  {\baselineskip}%
  {\baselineskip}{\Large\color{black}}}%
\renewcommand{\subsubsection}{\@startsection{subsubsection}{3}{0mm}
  {\baselineskip}%
  {\baselineskip}{\large\color{black}}}%
\makeatother

%%%%%%%%%%%%%%%%%%%%%%%%%%%%%%%%%%%%%%%%%%%%%%%%%%%%%%%%%%%%%%%%%%%%%%%%%%%%%%
% Table of contents styling
%
% Finamore: added dotted lines also for chapters
% solution found at http://tex.stackexchange.com/questions/62438/how-to-add-leaders-to-table-of-contents-without-tocloft
\makeatletter
\renewcommand*\l@chapter[2]{%
    \ifnum \c@tocdepth >\m@ne
        \addpenalty{-\@highpenalty}%
        \vskip 1.0em \@plus\p@
        \setlength\@tempdima{1.5em}%
        \begingroup
        \parindent \z@ \rightskip \@pnumwidth
        \parfillskip -\@pnumwidth
        \leavevmode %\bfseries
        \advance\leftskip\@tempdima
        \hskip -\leftskip
        #1\nobreak
        \xleaders\hbox{$\m@th
        \mkern \@dotsep mu\hbox{.}\mkern \@dotsep mu$}\hfill%
        \nobreak\hb@xt@\@pnumwidth{\hss #2}\par
        \penalty\@highpenalty
        \endgroup
    \fi}
\renewcommand*\l@section{\@dottedtocline{1}{1.5em}{2.3em}}
\renewcommand*\l@subsection{\@dottedtocline{2}{3.8em}{3.2em}}
\renewcommand*\l@subsubsection{\@dottedtocline{3}{7.0em}{4.1em}}
\renewcommand*\l@paragraph{\@dottedtocline{4}{10em}{5em}}
\renewcommand*\l@subparagraph{\@dottedtocline{5}{12em}{6em}}
\renewcommand*{\@dotsep}{1}
\makeatother

%%%%%%%%%%%%%%%%%%%%%%%%%%%%%%%%%%%%%%%%%%%%%%%%%%%%%%%%%%%%%%%%%%%%%%%%%%%%%%
% Header and footer definitions
%

\setlength{\headheight}{52pt}
\setlength{\footskip}{47pt}
\renewcommand{\headrulewidth}{0pt}
\pagestyle{fancyplain}

%\fancyhead[L]{\textsf{\\ D\deliverablenumber ~\deliverabletitleshort}}
%\fancyhead[R]{\setlength{\unitlength}{1mm}
%\begin{picture}(0,0)
%  \put(-70,-4){\includegraphics[width=70mm]{logos/logo.png}}
%\end{picture}
%}
%\fancyfoot[L]{\setlength{\unitlength}{1mm}
%\begin{picture}(0,0)
%  \put(0,2){\includegraphics[width=15mm]{logos/logo-short.pdf}}
%\end{picture}
%}
\fancyfoot[C]{\vspace{-10mm}\textsf{\thepage\hspace{0.3em}of \pageref*{LastPage}}}
%\fancyfoot[R]{\vspace{-10mm}\textsf{Revision \deliverablerevision\hspace{0.3em}of \deliverablerevdate}}

%%%%%%%%%%%%%%%%%%%%%%%%%%%%%%%%%%%%%%%%%%%%%%%%%%%%%%%%%%%%%%%%%%%%%%%%%%%%%%
% Miscellaneous document adjustments go here

% Call the bibilography a references section
\renewcommand*{\bibname}{References}

% Number figures and tables globally
\counterwithout{figure}{chapter}
\counterwithout{table}{chapter}

%%%%%%%%%%%%%%%%%%%%%%%%%%%%%%%%%%%%%%%%%%%%%%%%%%%%%%%%%%%%%%%%%%%%%%%%%%%%%%
% Define some colors

\definecolor{darkred}{RGB}{128, 0, 0}
\definecolor{darkgray}{RGB}{95, 95, 95}

\definecolor{architecuregreen}{RGB}{135, 222, 170}
\definecolor{experimentationblue}{RGB}{128, 179, 255}
\definecolor{measurementred}{RGB}{255, 128, 128}

%%%%%%%%%%%%%%%%%%%%%%%%%%%%%%%%%%%%%%%%%%%%%%%%%%%%%%%%%%%%%%%%%%%%%%%%%%%%%%
% Define some useful commands
 %\EDNOTE{who}{what} - place an editor's note in the document.
 %\INNOTE{who}{what} - place an inline note in the document

\newcommand{\EDNOTE}[2]{
  \par\large\centerline{
  \mbox{
  \begin{tabular}{ll}
    \textsf{\textbf{#1}} &
    \fbox{
    \begin{minipage}{0.6\linewidth}
      \em\color{red} #2
    \end{minipage}
    }
  \end{tabular}
  }}
  \par\vspace{4mm}
}

\newcommand{\INNOTE}[2]{[\textsf{\textbf{#1}}: {\em\color{red} #2}]}

%%%%%%%%%%%%%%%%%%%%%%%%%%%%%%%%%%%%%%%%%%%%%%%%%%%%%%%%%%%%%%%%%%%%%%%%%%%%%%
% Front page
%
\begin{document}
\thispagestyle{empty}
\parindent0pt

\newcolumntype{R}[1]{>{\raggedleft}p{#1}}

\vspace{1mm}

\vspace{50mm}

\begin{centering}
\centerline{\huge\em\bf\color{darkred}\deliverabletitle}

\vspace{2mm}
\vfill
\fbox{
  \begin{minipage}{\linewidth}
      \large\textsf{
        \begin{tabular}{lll}
	  \textbf{Author(s):}
	  \deliverableauthorlist
	\end{tabular}}
  \end{minipage}
}
 
 \end{centering}

%%%%%%%%%%%%%%%%%%%%%%%%%%%%%%%%%%%%%%%%%%%%%%%%%%%%%%%%%%%%%%%%%%%%%%%%%%%%%%
% Table of contents
%
\clearpage
\tableofcontents


%%%%%%%%%%%%%%%%%%%%%%%%%%%%%%%%%%%%%%%%%%%%%%%%%%%%%%%%%%%%%%%%%%%%%%%%%%%%%%
%%%%%%%%%%%%%%%%%%%%%%%%%%%%%%%%%%%%%%%%%%%%%%%%%%%%%%%%%%%%%%%%%%%%%%%%%%%%%%
%% CONTENT BEGINS HERE - YOU CAN START EDITING NOW
%%%%%%%%%%%%%%%%%%%%%%%%%%%%%%%%%%%%%%%%%%%%%%%%%%%%%%%%%%%%%%%%%%%%%%%%%%%%%%
%%%%%%%%%%%%%%%%%%%%%%%%%%%%%%%%%%%%%%%%%%%%%%%%%%%%%%%%%%%%%%%%%%%%%%%%%%%%%%

\chapter{mmb}

mmb (modular middlebox) is a vpp plugin that performs various
middlebox behaviors.

\chapter{mmb CLI guide}

\texttt{mmb <command>}\\
\textbf{SYNTAX :} \texttt{enable|disable|add|add-stateless|add-stateful|del|list|flush|show}

This parameter determines the command applied on the rule list.\\
Allowed values:
\begin{itemize}
\item \texttt{enable} : enable mmb on a given interface
\item \texttt{disable} : disable mmb on a given interface
\item \texttt{add}/\texttt{add-stateless} : add a stateless rule
\item \texttt{add-stateful} : add a stateful rule
\item \texttt{del} : remove a rule
\item \texttt{list} : list the rules
\item \texttt{show} : display informations about mmb
\item \texttt{flush} : remove all rules
\end{itemize}


\section{add rules}
%\texttt{[mod [<field>] [<value>]]|[strip [<field>]]|[drop]}
% \texttt{<field> [[<cond>] <value>] [<field> [[<cond>] <value>] ...]}
\texttt{mmb <add-keyword> <match> [<match> ...] <target> [<target> ...]}
\begin{itemize}

\item \texttt{<add-keyword>}\\
   \textbf{SYNTAX :} \texttt{add|add-stateless|add-stateful}

   The keyword determines if mmb has to keep track of connections that matches
   a given rule or not. \texttt{add}/\texttt{add-stateless} rules apply their
   targets at the packet level, each packet has to match the rule in order
   to apply the targets. \texttt{add-stateful} rules apply their targets at the
   connection/flow level. The \texttt{<match>} list of a stateful rule is used
   to add entries to the connection table. Once a connection is added to the table,
   the \texttt{<target>} list is applied to all packets of the connection, even if
   they don't match the rule. Additionally, \texttt{add-stateful} allows for
   special targets \texttt{map} and \texttt{shuffle}.

\item \texttt{<match>} \\
   \textbf{SYNTAX :} \texttt{[!]\ <field> [[<cond>] <value>]}

   This parameter is a constraint that determines the packets on which the
   rule will operate.

   If a \texttt{<match>} is composed of a \texttt{<field>} alone, the constraint
   is that the packet should contain the field. If it is composed of a
   \texttt{<field>} and a \texttt{<value>}, the constraint is that the packet
   should contain the field and it should be set to the specified value. If it
   is composed of a \texttt{<field>}, a \texttt{<cond>} and a \texttt{<value>},
   then the constraint is that the packet should contain the field, and the
   condition on the value should be true.

   The !\ operator applied on one constraint performs the logical NOT of
   the constraint. Multiple constraints can be inputted for the same rule, the
   resulting constraint is the logical AND of all inputted constraints.

   %The
   %\texttt{-r} argument performs the logical NOT of the logical AND of all
   %constraints (it reverses the matching constraint).

\item \texttt{<target>}  \\
   \textbf{SYNTAX :} \texttt{mod [...]|strip [...]|add [...]|drop [...]|map [...]|shuffle [...]}

   This parameter determines the action(s) to apply on matched packets.
   \begin{itemize}
   \item \texttt{mod <field> <value>}

      Modify a field on a packet.
   %TODO: if the field is omitted, default to matching field
   % if the value is omitted, default to 0
   \item \texttt{add <field> <value>}

      Add a tcp-opt to the packet.

   \item \texttt{strip [!]\ <field>}

      Strip options from a packet.

   If the !\ operator is placed after the strip keyword, the following option
   will be added to the whitelist (the only authorized options), if not it
   will be added to the blacklist (the forbidden options).

   The special keyword \texttt{all} can be used in a \texttt{strip} target
   to strip all options from the matched packet.
   % if the field is omitted, default to matching option
   \item \texttt{drop [<rate>]}

      Drop a packet with optional probability \texttt{<rate>} given in percentage,
      with a maximal precision of 0.01\%. The default rate is 100\%.

   \item \texttt{map <field> <value>}

      Perform a bidirectionnal mapping of the given \texttt{<field>} to a given value.
      Valid fields are: \texttt{ip-saddr}, \texttt{ip-daddr}, \texttt{ip6-saddr},
      \texttt{ip6-daddr}, \texttt{tcp-sport},
      \texttt{tcp-dport}, \texttt{udp-sport}, \texttt{udp-dport}, \texttt{ip-id},
      \texttt{ip6-flow-label}.

   \item \texttt{shuffle <field>}

      Perform a bidirectionnal mapping of the given \texttt{<field>} to a random value.
      Valid fields are: \texttt{tcp-seq-num}, \texttt{tcp-ack-num}, \texttt{tcp-sport},
      \texttt{tcp-dport}, \texttt{udp-sport}, \texttt{udp-dport}, \texttt{ip-id},
      \texttt{ip6-flow-label}.

   \end{itemize}

\end{itemize}

\subsection{\texttt{<cond>}}

A condition is applied on a \texttt{<value>} from a \texttt{<field>} to form
a constraint.
%Note that \texttt{>} and \texttt{<} have to be escaped if typed
%from a bash shell.
\\
Available conditions: \texttt{==}%, \texttt{!=}, \texttt{<=}, \texttt{>=},
%\texttt{<}, \texttt{>}.

\subsection{\texttt{<field>}}

Available fields:
   \begin{itemize}
   \item interfaces: \texttt{in}, \texttt{out}

   \item IPv4 fields: \texttt{ip-ver}, \texttt{ip-ihl},  \texttt{ip-dscp},
     \texttt{ip-ecn}, \texttt{ip-non-ect}\footnote{\label{note-ecn}not followed
     by \texttt{<value>}}, \texttt{ip-ect0}\textsuperscript{\ref{note-ecn}},
     \texttt{ip-ect1}\textsuperscript{\ref{note-ecn}},
     \texttt{ip-ce}\textsuperscript{\ref{note-ecn}}, \texttt{ip-len},
     \texttt{ip-id}, \texttt{ip-flags}, \texttt{ip-res}, \texttt{ip-df},
     \texttt{ip-mf}, \texttt{ip-frag-offset}, \texttt{ip-ttl}, \texttt{ip-proto},
     \texttt{ip-checksum}, \texttt{ip-saddr}\footnote{\label{note-prefix} the \texttt{<value>}
     can include a subnet mask, the \texttt{==} condition will become subnet matching},
     \texttt{ip-daddr}\textsuperscript{\ref{note-prefix}}.
    %TODO(\texttt{ip-opt}).

   \item IPv6 fields: \texttt{ip6-ver}, \texttt{ip6-traffic-class},
      \texttt{ip6-flow-label}, \texttt{ip6-len}, \texttt{ip6-next},
      \texttt{ip6-hop-limit}, \texttt{ip6-saddr}\textsuperscript{\ref{note-prefix}},
      \texttt{ip6-daddr}\textsuperscript{\ref{note-prefix}}, \texttt{ip6-payload}.

   \item ICMPv4 fields: \texttt{icmp-type}, \texttt{icmp-code}, \texttt{icmp-checksum},
     \texttt{icmp-payload}.

   \item User Datagram Protocol (UDP) fields: \texttt{udp-sport}, \texttt{udp-dport}, \texttt{udp-len},
     \texttt{udp-checksum}, \texttt{udp-payload}.

   \item Transmission Control Protocol (TCP) fields: \texttt{tcp-sport}, \texttt{tcp-dport}, \texttt{tcp-seq-num},
      \texttt{tcp-ack-num}, \texttt{tcp-offset}, \texttt{tcp-res}, \texttt{tcp-cwr},
      \texttt{tcp-ece}, \texttt{tcp-urg}, \texttt{tcp-ack}, \texttt{tcp-push},
      \texttt{tcp-res}, \texttt{tcp-syn}, \texttt{tcp-fin}, \texttt{tcp-flags},
      \texttt{tcp-win}, \texttt{tcp-checksum}, \texttt{tcp-urg-ptr}, \texttt{tcp-payload}.

   \item TCP options:
      \texttt{tcp-opt-mss}, \texttt{tcp-opt-wscale}, \texttt{tcp-opt-sackp},
      \texttt{tcp-opt-sack}, \texttt{tcp-opt-timestamp}, \texttt{tcp-opt-fast-open},
      \texttt{tcp-opt-fast-open}.

   \item Custom TCP options: \texttt{tcp-opt [<kind>]} \\
        Replace \texttt{<kind>} with the kind of option in decimal.
        When employed in a \texttt{<match>} without a \texttt{<kind>}, checks if the
        packet contains any option.

   \end{itemize}

\subsection{\texttt{<value>}}

The value of a field is in decimal or in hexadecimal if preceeded by \texttt{x}.
String value \texttt{ip}, \texttt{tcp}, \texttt{udp} and \texttt{icmp}
can be used with fields \texttt{net-proto} or \texttt{ip-proto}.

\subsection{stateful polices}

\section{Remove rules}

 \begin{itemize}
   \item \texttt{del}\\
         \textbf{SYNTAX :} \texttt{mmb del <rule-index>}

         Delete rule at given index and associated entries in connections tables.
   \item \texttt{flush}\\
         \textbf{SYNTAX :} \texttt{mmb flush}

         Delete all rules and entries in the connection table.
 \end{itemize}

\section{Display informations}

 \begin{itemize}
   \item \texttt{list}\\
         \textbf{SYNTAX :} \texttt{mmb list}

         List all rules.
   \item \texttt{show tables}\\
         \textbf{SYNTAX :} \texttt{mmb show tables [verbose]}

         Display informations about classifier tables such as masks, keys,
         capacity, and more.
   \item \texttt{show connections}\\
         \textbf{SYNTAX :} \texttt{mmb show connections [verbose]}

         Display informations about active connections used by stateful rules
         such as 5-tuples, connection type, expiring time, and more.
 \end{itemize}

\chapter{Examples}

\texttt{vpp\# mmb add all mod ip-ecn 0} \\
   ECN bleaching

\texttt{vpp\# mmb add ip-proto udp drop} \\
   Block UDP

\texttt{vpp\# mmb add ip-proto != tcp drop} \\
   Block every IP protocol but TCP

\texttt{vpp\# mmb add ip-proto != udp ip-proto != tcp drop} \\
   Block every IP protocol but TCP and UDP

\texttt{vpp\# mmb add ip-proto udp drop} \\
\texttt{vpp\# mmb add ip-proto tcp drop} \\
   Block TCP and UDP

\texttt{vpp\# mmb add tcp-dport 80 mod tcp-dport 443} \\
   Rewrite TCP port 80 to port 443

\texttt{vpp\# mmb add tcp-opt-mss strip tcp-opt-mss} \\
   Strip MSS option

\texttt{vpp\# mmb add tcp-opt-mss > 1500 mod tcp-opt-mss 1460} \\
   If MSS is larger than 1500, set it to 1460

\texttt{vpp\# mmb add tcp-opt strip !\ tcp-opt-mss} \\
   Strip all options but MSS

\texttt{vpp\# mmb add tcp-opt strip tcp-opt-mss strip tcp-opt-wscale} \\
   Strip MSS and WScale

\texttt{vpp\# mmb add tcp-opt-timestamp strip all} \\
   Strip all options if packet contains timestamp option

\texttt{vpp\# mmb add tcp-opt strip !\ tcp-opt-mss strip !\ tcp-opt-wscale} \\
   Strip all options except mss and wscale if packet contains timestamp option (whitelist)

\texttt{vpp\# mmb add tcp-opt strip tcp-opt-mss strip tcp-opt-wscale} \\
   Strip all mss and wscale if packet contains timestamp option (blacklist)

\texttt{vpp\# mmb add tcp-opt !\ tcp-opt-mss !\ tcp-opt-wscale  drop} \\
   Drop all TCP packets with options different than MSS or WScale.

\texttt{vpp\# mmb add !\ tcp-opt-mss !\ tcp-opt-wscale drop} \\
   Drop all TCP packets that do not contain MSS nor WScale.

\texttt{vpp\# mmb add tcp-opt 22 drop} \\
   Drop all TCP packets that contain option 22

\texttt{vpp\# mmb add-stateful ip-proto tcp ip-saddr 10.0.0.10/24 accept} \\
\texttt{vpp\# mmb add ip-proto tcp drop} \\
   Reflexive ACL that blocks everything but tcp connections initated from 10.0.0.10/24

\texttt{vpp\# mmb add-stateful ip-proto tcp tcp-saddr 10.0.0.10/24 map ip-saddr 30.30.30.30 shuffle tcp-sport} \\
   Source NAT

%%%%%%%%%%%%%%%%%%%%%%%%%%%%%%%%%%%%%%%%%%%%%%%%%%%%%%%%%%%%%%%%%%%%%%%%%%%%%%
%%%%%%%%%%%%%%%%%%%%%%%%%%%%%%%%%%%%%%%%%%%%%%%%%%%%%%%%%%%%%%%%%%%%%%%%%%%%%%
%% CONTENT ENDS HERE
%%%%%%%%%%%%%%%%%%%%%%%%%%%%%%%%%%%%%%%%%%%%%%%%%%%%%%%%%%%%%%%%%%%%%%%%%%%%%%
%%%%%%%%%%%%%%%%%%%%%%%%%%%%%%%%%%%%%%%%%%%%%%%%%%%%%%%%%%%%%%%%%%%%%%%%%%%%%%

\bibliographystyle{abbrv}
\small

% Bibtex files that contain the references. Use it as follows, omitting the
% file extension
%\bibliography{file1,file2,file3}
\end{document}
